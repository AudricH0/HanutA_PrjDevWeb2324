% Redéfinition du format de la commande \part pour éviter l'ajout d'une nouvelle page
\makeatletter
\def\@endpart{\vfil\newpage}
\makeatother

% Packages
\usepackage[utf8]{inputenc} % Encodage du texte en UTF-8
\usepackage[T1]{fontenc} % Encodage des caractères en sortie
\usepackage[french]{babel} % Support de la langue française
\usepackage{graphicx} % Inclusion d'images
\usepackage{setspace} % Espacement
\usepackage{titlesec} % Personnalisation des titres
\usepackage{fancyhdr} % En-têtes et pieds de page personnalisés
\usepackage{lipsum} % Génération de texte fictif Lorem Ipsum
\usepackage{amsmath} % Pour la commande \numberwithin
\usepackage{url}
\usepackage{enumitem} % Pour personnaliser les listes
\usepackage{adjustbox}
\usepackage{float} % Pour utiliser l'option [H]


% Marges du document
\usepackage[left=2.5cm,right=2.5cm,top=2.5cm,bottom=2.5cm]{geometry}
\setlength{\tabcolsep}{0.3cm} 

% En-têtes et pieds de page
\pagestyle{fancy}
\fancyhf{}
\rhead{\thepage}
\lhead{\leftmark}
\renewcommand{\headrulewidth}{0.5pt}
\renewcommand{\footrulewidth}{0pt}
\fancyhead[LE,RO]{\thepage}
\fancyhead[RE]{\leftmark}
\fancyhead[LO]{\rightmark}

\renewcommand{\chaptermark}[1]{\markboth{\thechapter.\ #1}{}}

% Numérotation des sections
\numberwithin{section}{chapter}

% Supprimer le préfixe "Chapitre X" des titres de chapitre dans la table des matières
\titleformat{\chapter}[display]
{\normalfont\huge\bfseries}{}{0pt}{\Huge}

\titleformat{\chapter}[display]
{\normalfont\huge\bfseries}{}{-10pt}{\Huge}

\titlespacing{\chapter}{0pt}{-32pt}{1cm}