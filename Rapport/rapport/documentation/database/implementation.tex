\newpage
\subsection{Implémentation}

J'ai utilisé l'outil de migration de Laravel pour créer la base de données. Voici les fichiers de migration utilisés : \\

\begin{table}[H]
	\begin{tabular}{|l|l|}
		\hline
		\textbf{Table} & \textbf{Fichier source} \\
		\hline
		Users &	\url{database/migrations/2014_10_12_000000_create_users_table.php} \\
		Class &	\url{database/migrations/2014_10_12_000000_create_class_table.php} \\
		Etuds &	\url{database/migrations/2014_10_12_000000_create_etuds_table.php} \\
		Eprs &	\url{database/migrations/2014_10_12_000000_create_eprs_table.php} \\
		Inscrs &	\url{database/migrations/2014_10_12_000000_create_inscrs_table.php} \\
		\hline
	\end{tabular} 
	\caption{Listes des fichiers migrations utilisés}
\end{table}



\textbf{Pour la configuration et la connexion à la base de données dans le projet, j'ai utilisé les fichiers suivants : }

\begin{itemize}[label=$\bullet$]
	\item \url{.env}
	\begin{itemize}[label=$\bullet$]
		\item Configuration de l'accès à la base de données.
	\end{itemize}
	\item \url{app/Providers/AppServiceProvider.php}
	\begin{itemize}[label=$\bullet$]
		\item Définition de la longueur par défaut des chaînes de caractères dans la base de données.
	\end{itemize}
	\item \url{config/database.php}
	\begin{itemize}[label=$\bullet$]
		\item  Définition du moteur de stockage InnoDB.
	\end{itemize}
\end{itemize}